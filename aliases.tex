\makeatletter
\newcommand*{\addFileDependency}[1]{% argument=file name and extension
  \typeout{(#1)}
  \@addtofilelist{#1}
  \IfFileExists{#1}{}{\typeout{No file #1.}}
}
\makeatother
 
\newcommand*{\myexternaldocument}[1]{%
    \externaldocument{#1}%
    \addFileDependency{#1.tex}%
    \addFileDependency{#1.aux}%
}

% use \S for all references to all kinds of sections, and \P to paragraphs
% (sadly, we cannot use the simpler \crefname{} macro because it would insert a space after the symbol)
\crefformat{part}{\S#2#1#3}
\crefformat{chapter}{\S#2#1#3}
\crefformat{section}{\S#2#1#3}
\crefformat{subsection}{\S#2#1#3}
\crefformat{subsubsection}{\S#2#1#3}
\crefformat{paragraph}{\P#2#1#3}
\crefformat{subparagraph}{\P#2#1#3}
%\crefmultiformat{part}{\S#2#1#3}{ and~\S#2#1#3}{, \S#2#1#3}{, and~\S#2#1#3}
%\crefmultiformat{chapter}{\S#2#1#3}{ and~\S#2#1#3}{, \S#2#1#3}{, and~\S#2#1#3}
\crefmultiformat{section}{\S#2#1#3}{ and~\S#2#1#3}{, \S#2#1#3}{, and~\S#2#1#3}
\crefmultiformat{subsection}{\S#2#1#3}{ and~\S#2#1#3}{, \S#2#1#3}{, and~\S#2#1#3}
\crefmultiformat{subsubsection}{\S#2#1#3}{ and~\S#2#1#3}{, \S#2#1#3}{, and~\S#2#1#3}
\crefmultiformat{paragraph}{\P\P#2#1#3}{ and~#2#1#3}{, #2#1#3}{, and~#2#1#3}
\crefmultiformat{subparagraph}{\P\P#2#1#3}{ and~#2#1#3}{, #2#1#3}{, and~#2#1#3}
%\crefrangeformat{part}{\mbox{\S\S#3#1#4--#5#2#6}}
%\crefrangeformat{chapter}{\mbox{\S\S#3#1#4--#5#2#6}}
\crefrangeformat{section}{\mbox{\S\S#3#1#4--#5#2#6}}
\crefrangeformat{subsection}{\mbox{\S\S#3#1#4--#5#2#6}}
\crefrangeformat{subsubsection}{\mbox{\S\S#3#1#4--#5#2#6}}
\crefrangeformat{paragraph}{\mbox{\P\P#3#1#4--#5#2#6}}
\crefrangeformat{subparagraph}{\mbox{\P\P#3#1#4--#5#2#6}}
% for \label[appsec]{...}
\crefname{part}{Part}{Parts}
\Crefname{part}{Part}{Parts}
\crefname{chapter}{ch.}{ch.}
\Crefname{chapter}{Ch.}{Ch.}
\crefname{figure}{figure}{figures}
\crefname{subfigure}{figure}{figures}
\Crefname{subfigure}{Figure}{Figures}
\crefname{appsec}{appendix}{appendices}
\Crefname{appsec}{Appendix}{Appendices}
\crefname{algocf}{algorithm}{algorithms}
\Crefname{algocf}{Algorithm}{Algorithms}
\crefname{enums,enumsi}{example}{examples}
\Crefname{enums,enumsi}{Example}{Examples}
\crefname{}{example}{examples} % lingmacros \toplabel has no internal name for the kind of label
\Crefname{}{Example}{Examples}
\crefformat{enums}{(#2#1#3)}
\crefformat{enumsi}{(#2#1#3)}
\crefrangeformat{enums}{\mbox{(#3#1#4--#5#2#6)}}
\crefrangeformat{enumsi}{\mbox{(#3#1#4--#5#2#6)}}
\crefformat{}{(#2#1#3)}
\crefname{xnumi}{example}{examples} % gb4e
\crefname{xnumi}{example}{examples} % gb4e
\Crefname{xnumii}{Example}{Examples} % gb4e
\Crefname{xnumii}{Example}{Examples} % gb4e
\crefformat{xnumi}{(#2#1#3)} % gb4e
\crefformat{xnumii}{(#2#1#3)} % gb4e
\crefrangeformat{enums}{\mbox{(#3#1#4--#5#2#6)}}
\crefrangeformat{enumsi}{\mbox{(#3#1#4--#5#2#6)}}
\crefrangeformat{xnumi}{\mbox{(#3#1#4--#5#2#6)}} % gb4e
\crefrangeformat{xnumii}{\mbox{(#3#1#4--#5#2#6)}} % gb4e
\crefmultiformat{enumsi}{(#2#1#3}{, #2#1#3)}{, #2#1#3}{, #2#1#3)}
\crefmultiformat{xnumi}{(#2#1#3}{, #2#1#3)}{, #2#1#3}{, #2#1#3)} % gb4e
\crefmultiformat{xnumii}{(#2#1#3}{, #2#1#3)}{, #2#1#3}{, #2#1#3)} % gb4e
\crefrangemultiformat{enumsi}{(#3#1#4--#5#2#6}{, #3#1#4--#5#2#6)}{, #3#1#4--#5#2#6}{, #3#1#4--#5#2#6)}
\crefrangemultiformat{xnumi}{(#3#1#4--#5#2#6}{, #3#1#4--#5#2#6)}{, #3#1#4--#5#2#6}{, #3#1#4--#5#2#6)} % gb4e
\crefrangemultiformat{xnumii}{(#3#1#4--#5#2#6}{, #3#1#4--#5#2#6)}{, #3#1#4--#5#2#6}{, #3#1#4--#5#2#6)} % gb4e

\ifx\creflastconjunction\undefined%
\newcommand{\creflastconjunction}{, and\nobreakspace} % Oxford comma for lists
\else%
\renewcommand{\creflastconjunction}{, and\nobreakspace} % Oxford comma for lists
\fi%

\newcommand*{\Fullref}[1]{\hyperref[{#1}]{\Cref*{#1}: \nameref*{#1}}}
\newcommand*{\fullref}[1]{\hyperref[{#1}]{\cref*{#1}: \nameref{#1}}}
\newcommand{\fnref}[1]{fn.~\ref{#1}} % don't use \cref{} due to bug in (now out-of-date) cleveref package w.r.t. footnotes
\newcommand{\Fnref}[1]{Fn.~\ref{#1}}

\newcommand{\citeposs}[2][]{\citeauthor{#2}'s (\citeyear[#1]{#2})}
\newcommand{\Citeposs}[2][]{\Citeauthor{#2}'s (\citeyear[#1]{#2})}



\makeatletter
\renewcommand{\@BIBLABEL}{\@emptybiblabel}
\newcommand{\@emptybiblabel}[1]{}
\makeatother
\DeclareMathOperator*{\argmax}{arg\,max}

\newcommand{\ucca}[1]{\textcolor{gray}{\textbf{\textsf{#1}}}}
\newcommand{\sst}[1]{\textsc{#1}}
\newcommand{\lexcat}[1]{\textsl{#1}}


\definecolor{orange}{rgb}{1,0.5,0}
\definecolor{mdgreen}{rgb}{0.05,0.6,0.05}
\definecolor{Acolor}{HTML}{EC5D57} % poppy red
\definecolor{Pcolor}{HTML}{70BF41} % grass green
\definecolor{Scolor}{HTML}{51A7F9} % sky blue
\definecolor{Lcolor}{HTML}{B36AE2} % friendly purple
\definecolor{mdblue}{rgb}{0,0,0.7}
\definecolor{dkblue}{rgb}{0,0,0.5}
\definecolor{dkgray}{rgb}{0.3,0.3,0.3}
\definecolor{slate}{rgb}{0.25,0.25,0.4}
\definecolor{gray}{rgb}{0.5,0.5,0.5}
\definecolor{ltgray}{rgb}{0.7,0.7,0.7}
\definecolor{purple}{rgb}{0.7,0,1.0}
\definecolor{lavender}{rgb}{0.65,0.55,1.0}



\newcommand{\com}[1]{}
\newcommand{\oa}[1]{\textcolor{red}{[OA: #1]}}
\newcommand{\daniel}[1]{\footnote{\color{cyan}DH: #1}}
\newcommand{\nss}[1]{\textcolor{magenta}{[NSS: #1]}}

\newcommand{\finalversion}[1]{} % hide for now, restore for camera-ready

\hyphenation{English}
\hyphenation{PredPatt}
\hyphenation{WordNet}
\hyphenation{WordNets}
\hyphenation{FrameNet}
\hyphenation{SemCor}
\hyphenation{SemEval}
\hyphenation{ParsedSemCor}
\hyphenation{VerbNet}
\hyphenation{PennConverter}
\hyphenation{an-aly-sis}
\hyphenation{an-aly-ses}
\hyphenation{base-line}
\hyphenation{comb-over}
\hyphenation{de-ve-lop-ed}
\hyphenation{news-text}
\hyphenation{nomi-nal}
\hyphenation{per-cept}
\hyphenation{per-cepts}
\hyphenation{post-edit-ing}
\hyphenation{shriv-eled}
\hyphenation{Huddle-ston}
\hyphenation{par-ti-ci-pant}
\hyphenation{par-ti-ci-pants}
\hyphenation{par-ti-ci-pa-tion}

%\setlength{\belowcaptionskip}{-10pt}
%\setlength{\parskip}{-0.085mm}

%\lstset{basicstyle=\ttfamily}


\usetikzlibrary{shapes,shapes.misc}

% for confusion matrix
\newcommand{\ApplyGradient}[1]{%
  \pgfmathsetmacro{\PercentColor}{(#1-0)/7.97}%
  \pgfmathsetmacro{\PercentInverse}{ifthenelse(\PercentColor > 65, 0, 100)}%
  %\textcolor{black!\PercentColor}{#1}
  \edef\x{\noexpand\cellcolor{red!\PercentColor}}\x\textcolor{black!\PercentInverse}{#1}%
}
\newcolumntype{R}{>{\small\collectcell\ApplyGradient}{c}<{\endcollectcell}}
